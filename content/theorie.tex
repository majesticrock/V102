\section{Zielsetzung}
Ziel der Versuch ist es die elastischen Konstanten eines Körpers (hier: einer Kugel) zu bestimmen. Anschließend soll das magnetische
Moment eines Permanentmagneten unter Zuhilfenahme eines Helmholtzspulenpaares bestimmt werden.
\section{Theorie}
\label{sec:Theorie}
    \subsection{Normal- und Schubspannung}
    Wenn auf die Oberfläche eines Körpers eine Kraft $F$ wirkt, kann es dabei zu Verformungen und Volumenänderungen am Körper
    kommen. Hier soll $F$ eine Zugkraft sein, das heißt eine Kraft die an der Oberfläche des Körpers \"zieht\". Diese Kraft wird
    als Spannung bezeichnet, wobei die auf der Oberfläche senkrechtstehende Komponente als Normalspannung $\sigma$ bezeichnet wird
    und die Komponente welche parallel zur Oberfläche verläuft als Schubspannung $\tau$.
    Zu Beachten ist dabei aber auch, dass sich die an der Oberfläche wirkenden Kräfte auf den ganzen Körper auswirken.
    Desweiteren wird eine Deformation als elastisch bezeichnet, wenn sich der Körper nach Einwirken einer Kraft wieder in seinen
    ursprünglichen Zustand zurückversetzt, also seine ursprüngliche Form wieder annimmt. Dies gilt bei jedem Körper für einen 
    gewissen Bereich des Zusammenhanges zwischen Kraft $F$ und Deformation (siehe \autoref{sec:hook}). Allgemein lässt sich dieser 
    proportionale Zusammenhang mithilfe von Konstanten beschreiben. Auf die Definition und den Zusammenhang dieser Konstanten wird 
    in \autoref{sec:konstanten} näher eingegangen.
    \subsection{Hook'sches Gesetz}
    \label{sec:hook}
    TODO: Grafik zum Hook'schen Gesetz einfügen!!!!!!!!!!!!!!!!!!!

    Das Hook'sche Gesetz gilt solange der Körper nur einer geringen Spannung $\sigma$ ausgesetzt ist, so dass nur eine elatische 
    Deformation vorliegt. In \autoref{fig:hook} wird dies als Hook'scher Bereich bezeichnet. Der Aufbau eines Körpers wird dabei
    als Kristallgitter betrachtet, in dem sich die Atome beziehugsweise Moleküle in einem Abstand $r$ in einem erlektrostatischen
    Gleichgewicht befinden. Durch Einwirkung einer äußeren Kraft verschiebt sich dieses Gleichgewicht und der Körper deformiert.
    Solange die Spannungen nicht zu groß sind, ist diese Deformation reversibel, sodass sich ein proportionaler Zusammenhang zwischen
    ergibt:
    \begin{equation}
    \label{eqn:hook}
        \sigma = E \frac {\Delta L} {L},
    \end{equation}     
    wobei $\frac {\Delta L}{L}$ die relative Längenänderung der Körpers und E das Elastizitätsmodul, auf welches in \autoref{sec:konstanten}
    näher eingegangen wird, bezeichnet.
    Insgesamt sind für die Beschreibung des Zusammenhanges zwischen Spannung und Deformation in einem Kristallgitter 36 Konstanten
    notwendig, durch Symmetrien veringert sich diese Zahl allerdings erheblich. 
    \subsection{Elastische Konstanten isotroper Körper}
    \label{sec:konstanten}
