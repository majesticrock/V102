\section{Diskussion}
\label{sec:Diskussion}

In beiden Messreihen existiert eine gewisse Ungenaugikeit, da die Kugel und damit der Magnet nicht exakt ausgerichtet werden kann.
In beiden Fällen ist natürlich das Erdmagnetfeld vorhanden. Diese Abweichung wird jedoch als gering eingeordnet und in der Auswertung nicht beachtet.

Ein Pendeln der Vorrichtung verändert die Messwerte stark, da die Apperatur zur Zeitbestimmung sehr empfindlich ist. Dieses tritt schnell auf, wenn gegen den Tisch gestoßen wird oder Ähnliches, entsprechend konnte eine geringe Pendelbewegung nicht vermieden werden.
Gerade gegen Ende der zweiten Messreihe werden die Amplituden der Torsionsschwingung sehr gering. Dadurch wird der Effekt, der selbst durch minimales Pendeln erzeugt wird groß.
Teilweise werden auf der Uhr Werte angezeigt, die zunächst zu gering sind, allerdings läuft sie kurz darauf weiter.
Werte, bei deren Messung sich diese Eigenschaft aufgezeigt hat, werden nicht aufgenommen.

Mithilfe des Justierrades kann vorsichtig eine Schwingung mit geringer Amplitude angeregt werden. Dies funktioniert zuverlässig.

Alle Messwerte in einer Messreihe sind sehr ähnlich und daher ist der Fehler des Mittelwertes sehr gering, weshalb die endgültigen Messwerte auch mit einem geringen Fehler behaftet sind.

Da zur Längenmessung des Drahtes lediglich ein Maßband vorhanden ist und der Draht auch während dieser Messung in der Apperatur eingespannt ist, kann jenes nur ungefähr an den Draht gehalten werden, weshalb diese Messgröße mit einem verhältnismäßig großem Fehler betrachtet werden kann.

Die Dicke des Drahtes wird an einem anderen gleichen Draht mit einer Mikrometerschraube gemessen. 
Eventuelle vorherige Messungen an diesem können diesen verformt haben, weshalb dieser eine andere Dicke haben könnte als der eigentliche Draht.
Fehler bei der Bestimmung der Dicke des Drahtes gehen jedoch mit der vierten Potenz in den Wert für $G$ und damit in den Wert für $\mu$ und $Q$ ein.
Dies erklärt die große Abweichung von dem Literaturwert $G_\text{Lit} = 79.3$ GPa \cite{G-stahl} circa um den Faktor $1.85$. Wird der Literaturwert verwendet, so gilt $\mu_\text{lit} = 0.3241$ und $Q_\text{lit} = 199$ GPa.

Die Mikrometerschraube selbst ist einfach am Nonus abzulesen und der Fehler von ihr ist als gering einzuordnen.
