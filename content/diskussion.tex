\section{Diskussion}
\label{sec:Diskussion}

Die aus den Messungen gewonnen Werte mit ihren relativen Fehlern lauten:

\vspace{0.5em}
\centerline{$G = 1.4675 \cdot 10^{11} \frac{\symup{N}}{\symup{m}^2} \pm 0.08 \%$}

\vspace{0.4em}

\centerline{$\mu = -0.285 \pm 0.7\%$}

\vspace{0.4em}

\centerline{$Q = 4.56 \cdot 10^{10} \frac{\symup{N}}{\symup{m}^2} \pm 0.4 \%$}

\vspace{0.4em}

\centerline{$m = (6.6 \pm 0.06) \cdot 10^{-3}$ Am² $\pm 2.7 \%$}
\vspace{0.5em}

Der errechnete Wert für $G$ weicht von dem Literaturwert $G_\text{Lit} = 79.3$ GPa \cite{G-stahl} um $185 \%$ ab.
Der daraus errechnete Wert für $\mu$ ist entgegen der Literatur negativ.
Wird der Literaturwert für $G$ verwendet, so gilt $\mu_\text{lit} = 0.3241$ und $Q_\text{lit} = 199$ GPa.
Der Wert für $Q$ ist nur 22.9 \% des Literaturwertes.

In beiden Messreihen existiert eine gewisse Ungenaugikeit, da die Kugel und damit der Magnet nicht exakt ausgerichtet werden kann.
In beiden Fällen ist natürlich das Erdmagnetfeld vorhanden. Diese Abweichung wird jedoch als gering eingeordnet und in der Auswertung nicht beachtet.

Ein Pendeln der Vorrichtung sorgt dafür, dass die Uhr frühzeitig stoppt oder zurücksetzt, da die Apparatur zur Zeitbestimmung sehr empfindlich ist. Dieses tritt schnell auf, wenn gegen den Tisch gestoßen wird oder Ähnliches, entsprechend konnte eine geringe Pendelbewegung nicht vermieden werden.
Gerade gegen Ende der zweiten Messreihe werden die Amplituden der Torsionsschwingung sehr gering. Dadurch wird der Effekt, der selbst durch minimales Pendeln erzeugt wird, groß.
Werte, bei deren Messung sich diese Eigenschaft aufgezeigt hat, werden nicht aufgenommen und in der Auswertung nicht betrachtet.

Mithilfe des Justierrades kann vorsichtig eine Schwingung mit geringer Amplitude angeregt werden. Dies funktioniert zuverlässig.

Alle Messwerte in einer Messreihe sind sehr ähnlich und daher ist der Fehler des Mittelwertes sehr gering, weshalb die endgültigen Messwerte auch mit einem geringen Fehler behaftet sind.

Da zur Längenmessung des Drahtes lediglich ein Maßband vorhanden ist und der Draht auch während dieser Messung in der Apparatur eingespannt ist, kann jenes nur ungefähr an den Draht gehalten werden, weshalb diese Messgröße mit einem Fehler belastet ist.
Dieser kann allerdings auf weniger als 0.5 cm eingeordnet werden, was im Vergleich zu dem Messwert von 85 cm deutlich kleiner als 1 \% ist. Dieser wird in der Rechnung nicht weiter beachtet.

Die Dicke des Drahtes wird an einem anderen gleichen Draht mit einer Mikrometerschraube gemessen. 
Eventuelle vorherige Messungen an diesem können diesen verformt haben, weshalb dieser eine andere Dicke haben könnte als der eigentliche Draht.
Fehler bei der Bestimmung der Dicke des Drahtes gehen jedoch mit der vierten Potenz in den Wert für $G$ und damit in den Wert für $\mu$ und $Q$ ein.
Damit kann die sehr große Abweichung von den Literaturwerten erklärt werden.

Die Mikrometerschraube selbst ist einfach am Nonus abzulesen und der Fehler von ihr ist als gering einzuordnen.
