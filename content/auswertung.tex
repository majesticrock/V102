\section{Auswertung}
\label{sec:Auswertung}
\subsection{Bestimmung des Schubmoduls}

\begin{table}[!htp]
\centering
\caption{Periodendauern ohne Magnetfeld.}
\label{tab:periodendauer}
\begin{tabular}{S[table-format=2.3]}
\toprule
{$T$ / s} \\
\midrule
18.281 \\
18.270 \\
18.276 \\
18.291 \\
18.298 \\
18.287 \\
18.281 \\
18.270 \\
18.284 \\
18.283 \\
\bottomrule
\end{tabular}
\end{table}

Zur Berechnung des Schubmoduls $G$ werden diverse Daten benötigt. Zunächst wird die Periodendauer der Drehschwingung des Systems gemessen, diese sind in \autoref{tab:periodendauer} zu finden, und gemittelt.
Zu der Berechnung des Mittelwertes wird 

\begin{equation}
  \label{eqn:mittelwert}
  \overline{T} = \frac{1}{10} \sum_{i=1}^{10} T_i
\end{equation}

verwendet. Der Fehler des Mittelwertes berechnet sich mittels 

\begin{equation}
  \label{eqn:FehlerMittelwert}
  \Delta \overline{T} = \frac{1}{\sqrt{10 \cdot (10-1)}} \sqrt{ \sum_{i=1}^{10} (T_i - \overline{T})^2} = 9 \cdot 10^{-5} \symup{s}.
\end{equation}

Damit berechnet sich die gemittelte Periodendauer zu
\\ \\
\centerline{$\overline{T} = (18.28210 \pm 0.00009)$ s.}
\\ \\
Die Daten für die Kugel werden an der Apperatur abgelesen und sind in \autoref{tab:kugel} zu finden.

\begin{table}[!htp]
\centering
\caption{Eigenschaften der Kugel.}
\label{tab:kugel}
\begin{tabular}{S[table-format=3.1] @{${}\pm{}$} S[table-format=2.1] S[table-format=2.2] @{${}\pm{}$} S[table-format=1.2 S[table-format=2.1]}
\toprule
\multicolumn{2}{c}{$m_K$ / g} & \multicolumn{2}{c}{$2R_K$ / mm} & {$\theta_K$ / gcm²} \\
\midrule
583.3 & 23.5 & 51.03 & 2.04 & 22.5 \\
\bottomrule
\end{tabular}
\end{table}

Die Länge des Drahtes wird gemessen und beträgt $L = 0.85$m. Der Radius wird an zehn verschiedenen Stellen gemessen. Diese Werte in \autoref{tab:draht} zu finden.
Da die Werte bei jeder Messung dieselben sind, ist der gemittelte Wert von $R = 85$µm als fehlerfrei zu betrachten.

\begin{table}[!htp]
\centering
\caption{Eigenschaften des Drahtes.}
\label{tab:draht}
\begin{tabular}{S[table-format=2.0] S[table-format=1.3]}
\toprule
{$R$ / m $\cdot 10^\text{-6}$} & {$L$ / m} \\
\midrule
85 & 0.665 \\
\bottomrule
\end{tabular}
\end{table}

Damit lässt sich nach \eqref{eqn:schubmodul} das Schubmodul
\\ \\
\centerline{$G = (1.4675 \pm 0.0012) \cdot 10^{11} \frac{\symup{N}}{\symup{m}^2}$}
\\ \\
bestimmen. Der Fehler ergibt sich dabei nach Gauß zu

\begin{equation}
 \Delta G = \frac{16 \pi L}{5 R^4} \cdot \sqrt{ \bigg( \frac{2 m_K R_K^2}{\overline{T}^3} \cdot \Delta \overline{T} \bigg)^2 + \bigg( \frac{2 m_k R_k^2}{\overline{T}^2} \cdot \Delta R_K \bigg)^2 + \bigg( \frac{R_K^2}{\overline{T}^2} \cdot \Delta m_K \bigg)^2} = 1.2 \cdot 10^{8} \frac{\symup{N}}{\symup{m}^2}
\end{equation}

\subsection{Bestimmung der Querkontraktionszahl und des Kompressionsmoduls}

Die Querkontraktionszahl $\mu$ berechnet sich nach \eqref{eqn:mubestimmen}. Der zugehörige Fehler ergibt sich nach Gauß zu

\begin{equation}
  \Delta \mu = \frac{1}{2} \cdot \sqrt{ \bigg( \frac{1}{G} \cdot \Delta E \bigg)^2 + \bigg( \frac{E}{G^2} \cdot \Delta G \bigg)^2} = 0.002.
\end{equation}

Dazu wird zusätzlich der Wert für das Elastizitätsmodul 

\vspace{.5em}
\centerline{$E = (21.00 \pm 0.05) \cdot 10^{10} \frac{\symup{N}}{\symup{m}^2}$}
\vspace{.5em}

benötigt. Dieser und dessen zugehöriger Fehler wurden angegeben. Damit ergibt sich 

\centerline{$\mu = -0.285 \pm 0.002$.}
Das Kompressionsmodul $Q$ kann mit \eqref{eqn:Qbestimmen} berechnet werden und ist somit

\vspace{.5em}
\centerline{$Q = (4.56 \pm 0.02) \cdot 10^{10} \frac{\symup{N}}{\symup{m}^2}$.}
\vspace{.5em}

Der Fehler $\Delta Q$ ergibt sich dabei nach Gauß zu

\begin{equation}
  \Delta Q = \sqrt{ \bigg( \frac{G^2}{(E - 3G)^2} \cdot \Delta E \bigg)^2 + \bigg( \frac{E^2}{3(3G - E)^2} \cdot \Delta G \bigg)^2} = 2 \cdot 10^{8} \frac{\symup{N}}{\symup{m}^2}.
\end{equation}


\subsection{Bestimmung des magnetischen Momentes}

\begin{table}[!htp]
    \centering
    \caption{Messdaten mit eingeschaltetem Magnetfeld.}
    \label{tab:magnetfeld}
    \begin{subtable}{0.18\textwidth}
\centering
\subcaption{$I$ = 0.5 A}
\label{tab:magnet0-5}
\begin{tabular}{S[table-format=2.3]}
\toprule
{$T$ / s} \\
\midrule
11.558 \\
11.556 \\
11.549 \\
11.547 \\
11.540 \\
11.545 \\
11.524 \\
11.531 \\
11.528 \\
11.523 \\
\bottomrule
\end{tabular}
\end{subtable}
    \begin{subtable}{0.18\textwidth}
\centering
\subcaption{$I$ = 1.0 A}
\label{tab:magnet1-0}
\begin{tabular}{S[table-format=1.3]}
\toprule
{$T$ / s} \\
\midrule
9.024 \\
9.025 \\
9.013 \\
9.016 \\
9.007 \\
9.002 \\
9.012 \\
9.002 \\
8.996 \\
9.008 \\
\bottomrule
\end{tabular}
\end{subtable}
    \begin{subtable}{0.18\textwidth}
\centering
\subcaption{$I$ = 1.5 A}
\label{tab:magnet1-5}
\begin{tabular}{S[table-format=1.3]}
\toprule
{$T$ / s} \\
\midrule
7.543 \\
7.539 \\
7.541 \\
7.540 \\
7.543 \\
7.540 \\
7.531 \\
7.521 \\
7.520 \\
7.519 \\
\bottomrule
\end{tabular}
\end{subtable}
    \begin{subtable}{0.18\textwidth}
\centering
\subcaption{$I$ = 2.0 A}
\label{tab:magnet2-0}
\begin{tabular}{S[table-format=1.3]}
\toprule
{$T$ / s} \\
\midrule
6.591 \\
6.576 \\
6.590 \\
6.582 \\
6.583 \\
6.585 \\
6.567 \\
6.567 \\
6.574 \\
6.577 \\
\bottomrule
\end{tabular}
\end{subtable}
    \begin{subtable}{0.18\textwidth}
\centering
\subcaption{$I$ = 2.5 A}
\label{tab:magnet2-5}
\begin{tabular}{S[table-format=1.3]}
\toprule
{$T$ / s} \\
\midrule
5.940 \\
5.950 \\
5.934 \\
5.934 \\
5.933 \\
5.931 \\
5.935 \\
5.937 \\
5.934 \\
5.929 \\
\bottomrule
\end{tabular}
\end{subtable}
    
    \begin{subtable}{0.18\textwidth}
\centering
\subcaption{$I$ = 3.0 A}
\label{tab:magnet3-0}
\begin{tabular}{S[table-format=1.3]}
\toprule
{$T$ / s} \\
\midrule
5.445 \\
5.448 \\
5.444 \\
5.446 \\
5.438 \\
5.443 \\
5.446 \\
5.449 \\
5.442 \\
5.444 \\
\bottomrule
\end{tabular}
\end{subtable}
    \begin{subtable}{0.18\textwidth}
\centering
\subcaption{$I$ = 3.5 A}
\label{tab:magnet3-5}
\begin{tabular}{S[table-format=1.3]}
\toprule
{$T$ / s} \\
\midrule
5.060 \\
5.060 \\
5.057 \\
5.050 \\
5.057 \\
5.058 \\
5.057 \\
5.057 \\
5.050 \\
5.051 \\
\bottomrule
\end{tabular}
\end{subtable}
    \begin{subtable}{0.18\textwidth}
\centering
\subcaption{$I$ =  4.0 A}
\label{tab:magnet4-0}
\begin{tabular}{S[table-format=1.3]}
\toprule
{$T$ / s} \\
\midrule
4.749 \\
4.740 \\
4.737 \\
4.749 \\
4.750 \\
4.754 \\
4.752 \\
4.745 \\
4.754 \\
4.730 \\
\bottomrule
\end{tabular}
\end{subtable}
    \begin{subtable}{0.18\textwidth}
\centering
\subcaption{$I$ = 4.5 A}
\label{tab:magnet4-5}
\begin{tabular}{S[table-format=1.3]}
\toprule
{$T$ / s} \\
\midrule
4.461 \\
4.449 \\
4.452 \\
4.476 \\
4.466 \\
4.458 \\
4.479 \\
4.475 \\
4.472 \\
4.475 \\
\bottomrule
\end{tabular}
\end{subtable}
    \begin{subtable}{0.18\textwidth}
\centering
\subcaption{$I$ = 5.0 A}
\label{tab:magnet5-0}
\begin{tabular}{S[table-format=1.3]}
\toprule
{$T$ / s} \\
\midrule
4.243 \\
4.216 \\
4.244 \\
4.208 \\
4.201 \\
4.198 \\
4.202 \\
4.201 \\
4.283 \\
4.286 \\
\bottomrule
\end{tabular}
\end{subtable}
\end{table}

Die Werte in \autoref{tab:magnetfeld} werden zur Veranschaulichung in einem Plot aufgetragen. Die Werte werden hier mittels linearer Regression mit einem Graphen der Form $\frac{1}{T_m^2} = aB_H + b$ angenähert.
Dabei ist $B_H$ die magnetische Feldstärke in der Mitte der Helmholtzspule. Diese berechnet sich nach

\begin{equation}
    B_H = \bigg( \sqrt{\frac{4}{5}} \bigg)^3 \cdot \frac{\mu_0 N I}{R}.
\end{equation}

Dieser Plot ist in \autoref{fig:plot_magnet} zu sehen.

\begin{figure}
  \centering
  \includegraphics{plot-magnet.pdf}
  \caption{Plot und Fit der Periodendauern.}
  \label{fig:plot_magnet}
\end{figure}

Mittels Python 3.7.0 werden die Variablen als

\vspace{.5em}
\centerline{$a = (1.08 \pm 0.03)$ $\frac{1}{\symup{T} \symup{s}^2}$}

\centerline{$b = (1.63 \pm 0.08) \cdot 10^{-3}$ $\frac{1}{\symup{s}^2}$}
\vspace{.5em}

bestimmt. Des Weiteren folgt dem Verhältnis für $T_m$ und $m$ in \eqref{eqn:Periodendauermagnet} direkt

\begin{equation}
\label{eqn:verhalt}
  aB + b = \frac{mB_H}{4 \pi^2 \theta} + \frac{D}{4 \pi^2 \theta}.
\end{equation}

Das Gesamtträgheitsmoment $\theta$ ergibt sich nach Addition der Trägheitsmomente der Kugel $\theta_K$, welches sich nach \eqref{eqn:traegheit} berechnet, und des Trägheitsmomentes der Halterung $\theta_H$, welches an der Apperatur abgelesen wird.
Damit folgt der Zusammenhang

\begin{equation}
  \theta = \theta_H + \theta_K = \theta_H + \frac {2}{5} m_K R_K^{2}.
\end{equation}

Der zugehörige Fehler nach Gauß ergibt sich zu

\begin{equation}
  \Delta \theta = \frac{2}{5} \sqrt{ \bigg( R_K^2 \cdot \Delta m_K \bigg)^2 + \bigg( 2m_K R_K \cdot \Delta R_K \bigg)^2 } = 0.001 \cdot 10^{-4} \symup{kg} \symup{m}^2.
\end{equation}

Der angegebene Wert für das Trägheitsmoment der Halterung ist

\vspace{.5em}
\centerline{$\theta_H = 22.5 \cdot 10^{-7}$ kgm².}
\vspace{.5em}

Also ist

\vspace{.5em}
\centerline{$\theta = (1.555 \pm 0.001) \cdot 10^{-4}$ kgm².}
\vspace{.5em}

Das Verhältnis in \eqref{eqn:verhalt} muss für alle $B_H$ gelten, da das magnetische Moment $m$ des Permanentmangneten nicht von dem äußeren Magnetfeld abhängt.
Damit muss

\begin{equation}
  m = 4 \pi^2 \theta a
\end{equation}

gelten. Der Fehler nach Gauß berechnet sich zu

\vspace{.5em}
\centerline{$\Delta m = 4 \pi^2 \sqrt{ \bigg( \theta \cdot \Delta a \bigg)^2 + \bigg( a \cdot \Delta \theta \bigg)^2 } = 0.2 \cdot 10^{-3}$ Am².}
\vspace{.5em}

Somit ergibt sich

\vspace{.5em}
\centerline{$m = (6.6 \pm 0.2) \cdot 10^{-3}$ Am².}
\vspace{.5em}