\section{Auswertung}
\label{sec:Auswertung}
\subsection{Bestimmung des Schubmoduls}

\begin{table}[!htp]
\centering
\caption{Periodendauern ohne Magnetfeld.}
\label{tab:periodendauer}
\begin{tabular}{S[table-format=2.3]}
\toprule
{$T$ / s} \\
\midrule
18.281 \\
18.270 \\
18.276 \\
18.291 \\
18.298 \\
18.287 \\
18.281 \\
18.270 \\
18.284 \\
18.283 \\
\bottomrule
\end{tabular}
\end{table}

Zur Berechnung des Schubmoduls $G$ werden diverse Daten benötigt. Zunächst wird die Periodendauer der Drehschwingung des Systems gemessen, diese sind in \autoref{tab:periodendauer} zu finden, und mithilfe von \eqref{eqn:mittelwert} gemittelt. Der Mittelwert beträgt
\\ \\
\centerline{$\overline{T} = (18.28210 \pm 0.00009)$ s.}
\\ \\
Die Daten für die Kugel und den Draht sind in \autoref{tab:kugel} und \autoref{tab:draht} zu finden.

\begin{table}[!htp]
\centering
\caption{Eigenschaften der Kugel.}
\label{tab:kugel}
\begin{tabular}{S[table-format=3.1] @{${}\pm{}$} S[table-format=2.1] S[table-format=2.2] @{${}\pm{}$} S[table-format=1.2 S[table-format=2.1]}
\toprule
\multicolumn{2}{c}{$m_K$ / g} & \multicolumn{2}{c}{$2R_K$ / mm} & {$\theta_K$ / gcm²} \\
\midrule
583.3 & 23.5 & 51.03 & 2.04 & 22.5 \\
\bottomrule
\end{tabular}
\end{table}

\begin{table}[!htp]
\centering
\caption{Eigenschaften des Drahtes.}
\label{tab:draht}
\begin{tabular}{S[table-format=2.0] S[table-format=1.3]}
\toprule
{$R$ / m $\cdot 10^\text{-6}$} & {$L$ / m} \\
\midrule
85 & 0.665 \\
\bottomrule
\end{tabular}
\end{table}
%TODO: Gleichungsverweis
Damit lässt sich nach GLEICHUNG das Schubmodul
\\ \\
\centerline{$G = (1.4675 \pm 0.0012) \cdot 10^{11} \frac{\symup{N}}{\symup{m}^2}$}
\\ \\
Der Fehler ergibt sich nach \eqref{eqn:fehlerfortpflanzung}.

\subsection{Bestimmung der Querkontraktionszahl und des Kompressionsmoduls}

Die Querkontraktionszahl $\mu$ berechnet sich nach \eqref{eqn:mubestimmen}. Dazu wird zusätzlich der Wert für das Elastizitätsmodul 

\centerline{$E = (21.00 \pm 0.05) \cdot 10^{10} \frac{\symup{N}}{\symup{m}^2}$}
benötigt. Dieser wurde angegeben. Damit ergibt sich 

\centerline{$\mu = -0.285 \pm 0.002$.}
Das Kompressionsmodul $Q$ kann mit \eqref{eqn:Qbestimmen} berechnet werden und ist somit
\\ \\
\centerline{$Q = (4.56 \pm 0.02) \cdot 10^{10} \frac{\symup{N}}{\symup{m}^2}$.}
\\ \\
Zur Berechnung des Fehlers wird in beiden Fällen \eqref{eqn:fehlerfortpflanzung} verwendet.


\subsection{Bestimmung des magnetischen Momentes}

\begin{table}[!htp]
    \centering
    \caption{Messdaten mit eingeschaltetem Magnetfeld.}
    \label{tab:magnetfeld}
    \begin{subtable}{0.18\textwidth}
\centering
\subcaption{$I$ = 0.5 A}
\label{tab:magnet0-5}
\begin{tabular}{S[table-format=2.3]}
\toprule
{$T$ / s} \\
\midrule
11.558 \\
11.556 \\
11.549 \\
11.547 \\
11.540 \\
11.545 \\
11.524 \\
11.531 \\
11.528 \\
11.523 \\
\bottomrule
\end{tabular}
\end{subtable}
    \begin{subtable}{0.18\textwidth}
\centering
\subcaption{$I$ = 1.0 A}
\label{tab:magnet1-0}
\begin{tabular}{S[table-format=1.3]}
\toprule
{$T$ / s} \\
\midrule
9.024 \\
9.025 \\
9.013 \\
9.016 \\
9.007 \\
9.002 \\
9.012 \\
9.002 \\
8.996 \\
9.008 \\
\bottomrule
\end{tabular}
\end{subtable}
    \begin{subtable}{0.18\textwidth}
\centering
\subcaption{$I$ = 1.5 A}
\label{tab:magnet1-5}
\begin{tabular}{S[table-format=1.3]}
\toprule
{$T$ / s} \\
\midrule
7.543 \\
7.539 \\
7.541 \\
7.540 \\
7.543 \\
7.540 \\
7.531 \\
7.521 \\
7.520 \\
7.519 \\
\bottomrule
\end{tabular}
\end{subtable}
    \begin{subtable}{0.18\textwidth}
\centering
\subcaption{$I$ = 2.0 A}
\label{tab:magnet2-0}
\begin{tabular}{S[table-format=1.3]}
\toprule
{$T$ / s} \\
\midrule
6.591 \\
6.576 \\
6.590 \\
6.582 \\
6.583 \\
6.585 \\
6.567 \\
6.567 \\
6.574 \\
6.577 \\
\bottomrule
\end{tabular}
\end{subtable}
    \begin{subtable}{0.18\textwidth}
\centering
\subcaption{$I$ = 2.5 A}
\label{tab:magnet2-5}
\begin{tabular}{S[table-format=1.3]}
\toprule
{$T$ / s} \\
\midrule
5.940 \\
5.950 \\
5.934 \\
5.934 \\
5.933 \\
5.931 \\
5.935 \\
5.937 \\
5.934 \\
5.929 \\
\bottomrule
\end{tabular}
\end{subtable}
    
    \begin{subtable}{0.18\textwidth}
\centering
\subcaption{$I$ = 3.0 A}
\label{tab:magnet3-0}
\begin{tabular}{S[table-format=1.3]}
\toprule
{$T$ / s} \\
\midrule
5.445 \\
5.448 \\
5.444 \\
5.446 \\
5.438 \\
5.443 \\
5.446 \\
5.449 \\
5.442 \\
5.444 \\
\bottomrule
\end{tabular}
\end{subtable}
    \begin{subtable}{0.18\textwidth}
\centering
\subcaption{$I$ = 3.5 A}
\label{tab:magnet3-5}
\begin{tabular}{S[table-format=1.3]}
\toprule
{$T$ / s} \\
\midrule
5.060 \\
5.060 \\
5.057 \\
5.050 \\
5.057 \\
5.058 \\
5.057 \\
5.057 \\
5.050 \\
5.051 \\
\bottomrule
\end{tabular}
\end{subtable}
    \begin{subtable}{0.18\textwidth}
\centering
\subcaption{$I$ =  4.0 A}
\label{tab:magnet4-0}
\begin{tabular}{S[table-format=1.3]}
\toprule
{$T$ / s} \\
\midrule
4.749 \\
4.740 \\
4.737 \\
4.749 \\
4.750 \\
4.754 \\
4.752 \\
4.745 \\
4.754 \\
4.730 \\
\bottomrule
\end{tabular}
\end{subtable}
    \begin{subtable}{0.18\textwidth}
\centering
\subcaption{$I$ = 4.5 A}
\label{tab:magnet4-5}
\begin{tabular}{S[table-format=1.3]}
\toprule
{$T$ / s} \\
\midrule
4.461 \\
4.449 \\
4.452 \\
4.476 \\
4.466 \\
4.458 \\
4.479 \\
4.475 \\
4.472 \\
4.475 \\
\bottomrule
\end{tabular}
\end{subtable}
    \begin{subtable}{0.18\textwidth}
\centering
\subcaption{$I$ = 5.0 A}
\label{tab:magnet5-0}
\begin{tabular}{S[table-format=1.3]}
\toprule
{$T$ / s} \\
\midrule
4.243 \\
4.216 \\
4.244 \\
4.208 \\
4.201 \\
4.198 \\
4.202 \\
4.201 \\
4.283 \\
4.286 \\
\bottomrule
\end{tabular}
\end{subtable}
\end{table}

Die Werte in \autoref{tab:magnetfeld} werden nach \eqref{eqn:mittelwert} gemittelt. Somit entstehen die Werte in \autoref{tab:mittelwerte-magnet}.

\begin{table}[!htp]
\centering
\caption{Gemittelte Periodendauer mit Magnetfeld.}
\label{tab:mittelwerte-magnet}
\begin{tabular}{S[table-format=1.1] S[table-format=1.4] S[table-format=2.4] @{${}\pm{}$} S[table-format=1.4]}
\toprule
{$I$ / A} & {$B$ / T} & \multicolumn{2}{c}{$\overline{T} \pm \Delta \overline{T}$ / s} \\
\midrule
0.5 & 0.0005 & 11.5401 & 0.0018 \\
1.0 & 0.0010 &  9.0105 & 0.0003 \\
1.5 & 0.0015 &  7.5337 & 0.0015 \\
2.0 & 0.0020 &  6.5792 & 0.0002 \\
2.5 & 0.0025 &  5.9357 & 0.0007 \\
3.0 & 0.0030 &  5.4445 & 0.0001 \\
3.5 & 0.0035 &  5.0557 & 0.0005 \\
4.0 & 0.0040 &  4.7460 & 0.0017 \\
4.5 & 0.0045 &  4.4663 & 0.0009 \\
5.0 & 0.0050 &  4.2282 & 0.0061 \\
\bottomrule
\end{tabular}
\end{table}

%TODO: Gleichungsreferenz
GLEICHUNG kann umgeformt werden zu

\begin{equation}
  mB + D = \frac{4 \pi^2 \theta}{T^2} := \Gamma
\end{equation}

Die gemittelten Werte werden mittels Python 3.7.0 geplottet. Dabei wird $\Gamma$ gegen $B$ aufgetragen. Die besagte Feldstärke berechnet sich bei Helmholtzspulen nach

\begin{equation}
    B_H = \bigg( \sqrt{\frac{4}{5}} \bigg)^3 \cdot \frac{\mu_0 N}{R}.
\end{equation}

%TODO: Gleichungsreferenz
Zur Berechnung von $\theta$ wird das Trägheitsmoment der Kugel $\theta_K$, das sich nach GLEICHUNG berechnet, mit dem gegebenen Trägheitsmoment $\theta_H$ der Halterung addiert.
\\ \\
\centerline{$\theta_H = 22.5 \cdot 10^{-7} \symup{kg} \symup{m}^2$}

\centerline{$\theta_K = (1.532 \pm 0.001) \cdot 10^{-4} \symup{kg} \symup{m}^2$}

\centerline{$\theta = (1.555 \pm 0.001) \cdot 10^{-4} \symup{kg} \symup{m}^2$}

Über die Messwerte wird ein linearer Fit der Form $\Gamma = mB + D$ gelegt, der sich nach \eqref{eqn:a} und \eqref{eqn:b} berechnet.
Dieser ist in \autoref{fig:plot_magnet} zu sehen.

\begin{figure}
  \centering
  \includegraphics{plot-magnet.pdf}
  \caption{Plot und Fit der Mittelwerte der Periodendauern.}
  \label{fig:plot_magnet}
\end{figure}

Mittels Python 3.7.0 werden die Werte als
\\ \\
\centerline{$m = (66.1 \pm 0.4) \cdot 10^{-3} \symup{A}\symup{m}^2$}

und

\centerline{$D = (10 \pm 1) \cdot 10^{-6} \frac{\symup{kg} \symup{m}^2}{\symup{s}^2}$}

bestimmt.
Dabei ist $m$ das magnetische Moment.